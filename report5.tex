\documentclass[12pt]{jsarticle}
\usepackage{ascmac}
\begin{document}

\title{プログラミング2、レポート課題5: 「例外に慣れよう」}
\date{\today}
\author{145771D 綱島隆太}
\maketitle

\section{課題説明}

%{\large (改行) ~ } とすれば、文字を大きくする範囲を指定できる。

残念ながら教科書では例外処理のことを扱っていない。例外が発生するコードとAPIを通して、例外処理に慣れよう。


\section{結果と考察}
\subsection{結果}
\subsubsection{ステップ1}
https://github.com/takashift/Report5.git

\subsubsection{ステップ2}

 \begin{shadebox}
\begin{verbatim}

   ∧_∧
str→( ´∀`)< ぬるぽ

  ( ・∀・)   | | ガッ
 と    )    | |
   Y /ノ    人
    / )    <  >__Λ∩
  _/し' //. V`Д´)/ ←str
 (_フ彡        /
java.lang.NullPointerException
	at Main.main(Main.java:8)
	at sun.reflect.NativeMethodAccessorImpl.invoke0(Native Method)
	at sun.reflect.NativeMethodAccessorImpl.invoke(
	NativeMethodAccessorImpl.java:62)
	at sun.reflect.DelegatingMethodAccessorImpl.invoke(
	DelegatingMethodAccessorImpl.java:43)
	at java.lang.reflect.Method.invoke(Method.java:498)
	at com.intellij.rt.execution.application.AppMain.main(
	AppMain.java:147)

\end{verbatim}
\end{shadebox}
%行間を開ける
\vspace{10pt}

\subsubsection{ステップ3}


%\subsubsection*{計算例1}
\subsubsection{ステップ4}
\begin{shadebox}
\begin{verbatim}
    
\end{verbatim}
\end{shadebox}
%行間を開ける
\vspace{10pt}

\subsection{考察}
最初、Runできず困惑した。mainメソッドの引数のString[] argsが必須であることを知った。

\bigskip


%\section{その他}


%\bigskip

\end{document}